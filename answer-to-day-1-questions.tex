% Options for packages loaded elsewhere
\PassOptionsToPackage{unicode}{hyperref}
\PassOptionsToPackage{hyphens}{url}
\documentclass[
]{article}
\usepackage{xcolor}
\usepackage[margin=1in]{geometry}
\usepackage{amsmath,amssymb}
\setcounter{secnumdepth}{-\maxdimen} % remove section numbering
\usepackage{iftex}
\ifPDFTeX
  \usepackage[T1]{fontenc}
  \usepackage[utf8]{inputenc}
  \usepackage{textcomp} % provide euro and other symbols
\else % if luatex or xetex
  \usepackage{unicode-math} % this also loads fontspec
  \defaultfontfeatures{Scale=MatchLowercase}
  \defaultfontfeatures[\rmfamily]{Ligatures=TeX,Scale=1}
\fi
\usepackage{lmodern}
\ifPDFTeX\else
  % xetex/luatex font selection
\fi
% Use upquote if available, for straight quotes in verbatim environments
\IfFileExists{upquote.sty}{\usepackage{upquote}}{}
\IfFileExists{microtype.sty}{% use microtype if available
  \usepackage[]{microtype}
  \UseMicrotypeSet[protrusion]{basicmath} % disable protrusion for tt fonts
}{}
\makeatletter
\@ifundefined{KOMAClassName}{% if non-KOMA class
  \IfFileExists{parskip.sty}{%
    \usepackage{parskip}
  }{% else
    \setlength{\parindent}{0pt}
    \setlength{\parskip}{6pt plus 2pt minus 1pt}}
}{% if KOMA class
  \KOMAoptions{parskip=half}}
\makeatother
\usepackage{graphicx}
\makeatletter
\newsavebox\pandoc@box
\newcommand*\pandocbounded[1]{% scales image to fit in text height/width
  \sbox\pandoc@box{#1}%
  \Gscale@div\@tempa{\textheight}{\dimexpr\ht\pandoc@box+\dp\pandoc@box\relax}%
  \Gscale@div\@tempb{\linewidth}{\wd\pandoc@box}%
  \ifdim\@tempb\p@<\@tempa\p@\let\@tempa\@tempb\fi% select the smaller of both
  \ifdim\@tempa\p@<\p@\scalebox{\@tempa}{\usebox\pandoc@box}%
  \else\usebox{\pandoc@box}%
  \fi%
}
% Set default figure placement to htbp
\def\fps@figure{htbp}
\makeatother
\setlength{\emergencystretch}{3em} % prevent overfull lines
\providecommand{\tightlist}{%
  \setlength{\itemsep}{0pt}\setlength{\parskip}{0pt}}
\usepackage{bookmark}
\IfFileExists{xurl.sty}{\usepackage{xurl}}{} % add URL line breaks if available
\urlstyle{same}
\hypersetup{
  pdftitle={SCM Analysis Assignment Attempt},
  pdfauthor={Olatunbosun Oyindasola},
  hidelinks,
  pdfcreator={LaTeX via pandoc}}

\title{SCM Analysis Assignment Attempt}
\author{Olatunbosun Oyindasola}
\date{2025-11-12}

\begin{document}
\maketitle

\textbf{How to run this:} 1. Copy the code above into your script editor
(top-left pane) 2. Save it as ``Day01\_Inventory\_Analysis.R'' in your
folder 3. Click ``Run'' button or press Ctrl+Enter (Cmd+Enter on Mac) to
run line by line 4. OR: Click ``Source'' to run entire script at once

\textbf{12:50 - 1:00 (10 min): Document Your Work}

Create a text file ``Day01\_Results.txt'' and answer:

\begin{enumerate}
\def\labelenumi{\arabic{enumi}.}
\tightlist
\item
  What was your total inventory value? 88780
\item
  Which products need reordering? 0
\item
  What's your most valuable product? Laptop
\item
  One thing you learned about R today
\item
  One thing that confused you
\end{enumerate}

\begin{center}\rule{0.5\linewidth}{0.5pt}\end{center}

\subsection{WHAT YOU SHOULD HAVE NOW:}\label{what-you-should-have-now}

✅ R and RStudio installed\\
✅ Completed basic R commands in console\\
✅ Created your first R script\\
✅ Calculated inventory metrics\\
✅ Identified products to reorder\\
✅ Saved your script

\begin{center}\rule{0.5\linewidth}{0.5pt}\end{center}

\subsection{📝 YOUR DAY 1 REPORT TO ME (Reply with
this):}\label{your-day-1-report-to-me-reply-with-this}

``` DAY 1 COMPLETE ✅

Time Spent: {[}X minutes{]}

Results: - Total Inventory Value: \$88780 - Products to Reorder: 0 -
Most Valuable Product: {[}Product name{]}

Key Learning: I am learning to assign variables and use markdown to
export.

Challenge: So far there are no challenges because this is the beginning
and I am conversant with R.

Question: so far, no question for now

Tomorrow's Commitment: same time tomorrow.

\end{document}
